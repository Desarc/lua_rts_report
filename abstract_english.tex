\pagestyle{empty}
\begin{abstract}
\noindent
Working with resource-constrained smart devices can be a challenge, even for experienced developers. Applications have to be designed around the limited resources, often restricting the possible development choices. Most commonly, development on small devices is done in C, which is a low-level language requiring knowledge of concepts like memory management and pointers. For the less experienced developer, the relative difficulty of learning and using these concepts may provide a significant ``entrance cost'' to the world of embedded systems.

\noindent
As an alternative to working with C, one might try porting the \gls{vm} for a higher level language to the device in order to support development on a higher abstraction level, provided this \gls{vm} is lightweight enough. Among higher-level languages, Lua is considered to be both simple and lightweight, and some efforts have already been made towards porting it to various embedded devices.

\noindent
Taking the abstraction one step further, we can base the development of embedded applications on \glspl{fsm}. Modeling an application as a set of \glspl{fsm} also provides some additional advantages, such as making it easier to maintain and verify. We can even take it to the level of graphical modeling with automatic code generation. The proposed solution is then to create a runtime system framework for state machines in Lua, with the motivating factors being that it may be lightweight and portable enough to be a feasible and possibly even good option for application development on small devices.

\noindent
The process used in this project for determining whether Lua state machine systems are feasible on small devices consists of creating a simple framework prototype, and running it with some sample applications on a selected small device (microcontroller). Additionally, some testing on resource use and performance is done, to provide data for potential comparison with alternatives in future work.

\noindent
The results obtained show that even the very simple framework proposed uses a lot of memory, compared both to what is available on the device, and the same application running without the framework. It is possible to reduce the memory use some, but at the cost of other properties, like robustness. However, even with optimizations, the memory use appears to be too high for a complete production-quality framework to be feasible, given that it is written in pure Lua.

\noindent
With respect to performance, the framework does not generally impair the application. The processing overhead is only noticeable when the application consists of very small tasks that are performed very frequently, but these do not appear to be particularly realistic.

\noindent
From the results obtained, it seems that creating and using a framework for state machines in Lua is feasible, but not practical, especially when working on small devices. Lua does not perform as good as for example C in this environment, and in environments with more resources available, options like Java or Python are likely more practical. While relatively small and efficient, the resource use of the Lua framework is still too high for small devices. Additionally, standard Lua lacks support for some types of functionality a complete framework might require. It is possible to replace parts of a Lua framework with more efficient C code, making it a more feasible alternative for small devices and possibly adding missing functionality. One could even go as far as making a pure C state machine framework, providing support for state machine definitions in Lua, similar to how other applications like Wireshark utilize Lua support. The results obtained indicate that this is likely the better option.

\noindent
The experiments conducted in this project do however not provide a basis for \emph{definite} conclusions, but offers some indications, and creates a foundation for further work. More experiments should be done in order to verify the relevance of these results.

\end{abstract}