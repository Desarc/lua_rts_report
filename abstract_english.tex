\pagestyle{empty}
\begin{abstract}
\noindent
The field of smart devices and the \emph{Internet of Things} has received a lot of research attention in the past years, and development is still progressing quickly. As the area of applications expands and this concept becomes more pervasive, new problems and challenges appear, and must be solved.

\noindent
Often, it is difficult to get started with application development on small devices, since they usually have severe constraints on available resources, and generally no support from an operating system. Application development is usually done in low-level languages like C or even assembly, making this process a significant challenge for less experienced developers.

\noindent
Additionally, it may be a challenge to document and verify the correctness of an application, depending on its complexity. Abstractions in the form of finite-state machines or component collaborations are often used in order to describe the behavior of an application in terms of verifiable and maintainable models. In some cases, these models can even be used to automatically generate the implementation code, significantly reducing development effort and the room for errors.

\noindent
When these challenges provide a potentially high ``entrance cost'' for application development on small devices, it raises the question whether it is possible to develop methods or frameworks in order to reduce this cost. One proposed solution is to create a runtime system for state machines in the Lua programming language, which is the focus of this report. Lua is an interpreted, lightweight, and portable scripting language, well-documented and with low resource requirements compared to other similar languages. If it is feasible to develop and use a system like this, it should hopefully serve as an option for simple, maintainable and verifiable application development on small devices.

\noindent
The process used in this project for determining whether Lua state machine systems are feasible on small devices consists of creating a simple framework prototype, and running it with some sample applications on a selected small device (microcontroller). Additionally, some testing on resource use and performance is done, to provide data for potential comparison with alternatives in future work.

\noindent
The results obtained show that even the very simple framework proposed uses a lot of memory, compared both to what is available on the device, and the same application running without the framework. It is possible to reduce the memory use some, but at the cost of other properties, like robustness. However, even with optimizations, the memory use appears to be too high for a complete production-quality framework to be feasible, given that it is written in pure Lua.

\noindent
With respect to performance, the framework does not generally impair the application. The processing overhead is only noticeable when the application consists of very small tasks that are performed very frequently, but these do not appear to be particularly realistic.

\noindent
From the results obtained, it seems that creating and using a framework for state machines in Lua is feasible, but not practical, especially when working on small devices. Lua does not perform as good as for example C in this environment, and in environments with more resources available, options like Java or Python are likely more practical. While relatively small and efficient, the resource use of the Lua framework is still too high for small devices. Additionally, standard Lua lacks support for some types of functionality a complete framework might require. It is possible to replace parts of a Lua framework with more efficient C code, making it a more feasible alternative for small devices and possibly adding missing functionality. One could even go as far as making a pure C state machine framework, providing support for state machine definitions in Lua, similar to how other applications like Wireshark utilize Lua support. The results obtained indicate that this is likely the better option.

\noindent
The experiments conducted in this project are however not very conclusive, only offering indications and creating a foundation for further work. More experiments should be done in order to verify the relevance of these results.

\end{abstract}