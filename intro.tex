\chapter{Introduction}
\label{ch:intro}
The field of smart devices and the \emph{Internet of Things} has received a lot of research attention in the past years, and development is still progressing quickly. As the area of applications expands and this concept becomes more pervasive, new problems and challenges appear, and must be solved.

\noindent
Often, it is difficult to get started with application development on small devices, since they usually have severe constraints on available resources, and generally no support from an operating system. Application development is usually done in low-level languages like C or even assembly, making this process a significant challenge for less experienced developers. Can we do better?

\noindent
Additionally, it may be a challenge to document and verify the correctness of an application, depending on its complexity. Abstractions in the form of finite-state machines or component collaborations are often used in order to describe the behavior of an application in terms of verifiable and maintainable models. In some cases, these models can even be used to automatically generate the implementation code, significantly reducing development effort and the room for errors.

\section{Problem Description and Scope}
\label{sec:problem_scope}
When these challenges provide a potentially high ``entrance cost'' for application development on small devices, it raises the question whether it is possible to develop methods or frameworks in order to reduce this cost. We could for example base development on a higher-level language than C, offering more abstractions with respect to issues like memory management. Additionally, we can use concepts like state machines to develop applications on an even higher level of abstraction, making the development process more intuitive, and the application more maintainable and verifiable.

\noindent
With these points in mind, one proposed solution is to create a runtime system framework for state machines in the Lua programming language, which is the focus of this report. Lua is an interpreted, lightweight, and portable scripting language, well-documented and with low resource requirements compared to other similar languages. It offers automatic memory management and a simple syntax, making it relatively easy to learn and use.

\noindent
Additionally, with a runtime system framework for state machines provided, the developers may only need to specify their application in the form of a set of \glspl{fsm}. These \gls{fsm} implementations are then bundled with the framework code, and run on the desired platform, given that the standard Lua interpreter is available on this platform. If it is feasible to develop and use a framework like this, it should hopefully serve as an option for simple, maintainable and verifiable application development on small devices.

\noindent
The focus of this project is to determine if it is feasible to develop a state machine system in Lua, and use this in the context of embedded devices. This is done by creating a prototype runtime system and running some sample applications on a selected microcontroller, while doing some tests in order to measure resource use. The results are then evaluated with respect to feasibility and practicality. Some analysis is also done towards comparing the solution with other possible solutions, such as using other programming languages, in order to determine if they may provide better options. This latter analysis is however purely theoretical and speculative, no experiments with alternative solutions are conducted.

\section{Structure of the Report}
\label{sec:structure}
This project is a combination of a technical and a development assignment, covering some experimental work, theoretical evaluation, and documentation for a proposed prototype. The report has the following structure:

\noindent
Chapter~\ref{ch:background} begins by introducing some background material that is useful in order to understand the rest of the report. This includes a quick introduction to Lua as well as some notes on state machine systems and \gls{m2m}.

\noindent
Chapter~\ref{ch:related_work} lists and summarizes work done by other people that is related to this project. More specifically, some existing frameworks and tools for developing state machine systems are mentioned, as well as some previous attempts at using the Lua programming language in an embedded context.

\noindent
Chapter~\ref{ch:initial_analysis} describes the analysis done prior to the experimental work in this project, serving as a basis for discussion. It compares Lua to other relevant programming languages like C, Java and Python, and attempts to evaluate Lua in the context of state machine systems and embedded devices on a fairly superficial level.

\noindent
Chapter~\ref{ch:experimental_work} is the main part of this report, describing the experiments conducted and results obtained. The experimental process is rather iterative, with the next step generally depending on the results from the previous. Each part of Chapter~\ref{ch:experimental_work} thus provides the goal, methodology and relevant results for that particular step.

\noindent
Finally, Chapter~\ref{ch:discussion_conclusion} analyzes the results obtained in Chapter~\ref{ch:experimental_work}, compares with the initial analysis in Chapter~\ref{ch:initial_analysis}, and attempts to provide a conclusion with respect to the problem description. Chapter~\ref{ch:discussion_conclusion} also suggests some options for continuing the work initiated by this project.