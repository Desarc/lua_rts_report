\begin{appendices}

\chapter{Lua code for Runtime Support System}
\label{app:rts_code}

\begin{luacode}
-- inheritable template for a state machine implementation
StateMachine = {
	TERMINATE_SELF = -1,
	EXECUTE_TRANSITION = 0,
	DISCARD_EVENT = 1,
	TERMINATE_SYSTEM = 2,
}

function StateMachine.run()
	error("No coroutine created for this state machine!")
end

function StateMachine:fire()
	error("'fire' function not yet implemented for this state machine!")
end

function StateMachine:id()
	if self.data then
		return self.data.id
	else
		error("StateMachine has no ID!")
	end
end

function StateMachine:get_state()
	if self.data then
		return self.data.current_state
	else
		error("StateMachine has no state!")
	end
end

function StateMachine:set_state(state)
	if not self.data then
		self.data = {}
	end
	self.data.current_state = state
end

function StateMachine:new()
	local o = {}
	setmetatable(o, { __index = self })
	return o
end

return StateMachine
\end{luacode}

\begin{listing}[htp]
\begin{luacode}
-- structure of events passed between state machines
Event = {}

function Event:state_machine_id()
	return self.data.state_machine_id
end

function Event:type()
	return self.data.event_type
end

function Event:get_data()
	return self.data.user_data
end

function Event:new(state_machine_id, event_type, user_data)
	local o = {}
	setmetatable(o, { __index = self })
	o.data = {state_machine_id = state_machine_id, event_type = event_type,
						user_data = user_data}
	return o
end

return Event
\end{luacode}
\end{listing}

\begin{listing}[htp]
\begin{luacode}
-- structure of timer used to signal an event
Timer = {}

function Timer:expires()
	return self.data.expires
end

function Timer:state_machine_id()
	return self.data.state_machine_id
end

function Timer:event()
	return self.data.event
end

function Timer:id()
	return self.data.id
end

function Timer.time()
	return tmr.read(tmr.SYS_TIMER)
end

function Timer:new(expires, state_machine_id, event)
	local o = {}
	setmetatable(o, { __index = self })
	local id = state_machine_id .. expires
	o.data = {id = id, expires = self.time()+expires,
						state_machine_id = state_machine_id, event = event}
	return o
end

return Timer
\end{luacode}
\end{listing}

\begin{listing}[htp]
\begin{luacode}
-- the scheduler keeping track of active state machines, timers and events
Scheduler = {}

local function controller_time()
  return tmr.read(tmr.SYS_TIMER)
end

function Scheduler:add_state_machine(state_machine)
  state_machine.run = coroutine.create(state_machine.fire)
  self.state_machine_list[state_machine:id()] = state_machine
  print("State machine '"..state_machine:id().."' added to scheduler.")
end

function Scheduler:remove_state_machine(state_machine)
  if self.state_machine_list[state_machine:id()] then
    result = true -- return true if state machine existed
    print("State machine '"..state_machine:id().."' removed from scheduler.")
  end
    self.state_machine_list[state_machine:id()] = nil
  return result
end

function Scheduler:add_to_queue(event)
  table.insert(self.event_queue, event)
end

function Scheduler:get_next_event()
  return table.remove(self.event_queue, 1)
end

function Scheduler.time()
  return tmr.read(tmr.SYS_TIMER)
end

local function timers_cmp(t1, t2)
  if t1:expires() < t2:expires() then return true end
end

function Scheduler:add_timer(timer)
  if timer:expires() then
    table.insert(self.timers, timer)
    table.sort(self.timers, timers_cmp)
  end
end

function Scheduler:stop_timer(id)
  for k, v in self.timers do
    if v:id() == id then
      table.remove(self.timers, k)
      break
    end
  end
end

function Scheduler:check_timers()
  local now = self.time()
  if self.timers[1] then
    if self.timers[1]:expires() < now then return table.remove(self.timers, 1) end
  end
end

function Scheduler:set_active_event(event)
  self.active_event = event
end

\end{luacode}
\end{listing}

\begin{listing}[htp]
\begin{luacode}
-- Scheduler cont.
function Scheduler:get_active_event()
  local event = self.active_event
  self.active_event = nil
  return event
end

function Scheduler:check_active()
  if table.getn(self.timers) > 0 or table.getn(self.event_queue) > 0 then
    return true
  else
    return false
  end
end

function Scheduler:new(system_type)
  local o = {}
  setmetatable(o, { __index = self })
  o.state_machine_list = {}
  o.event_queue = {}
  o.timers = {}
  return o
end

function Scheduler:run()
  print("Scheduler running.")
  local success, status, state_machine
  local start = self.time()

  while(true) do		
    local timer = self:check_timers()
    if timer then
      state_machine = self.state_machine_list[timer:state_machine_id()]
      self:set_active_event(timer:event())
      success, status = coroutine.resume(state_machine.run, state_machine)
      if not success then
        print("Success: "..tostring(success)..", status: "..status)
        self:remove_state_machine(state_machine)
        break
      elseif status == StateMachine.TERMINATE_SYSTEM then
        break
      end
    end

    local event = self:get_next_event()
    if event then
      state_machine = self.state_machine_list[event:state_machine_id()]
      self:set_active_event(event)
      success, status = coroutine.resume(state_machine.run, state_machine)
      if not success then
        print("Success: "..tostring(success)..", status: "..status)
        self:remove_state_machine(state_machine)
        break
      elseif status == StateMachine.TERMINATE_SYSTEM then
        break
      end
    end
  end
  print("Terminating system...")
end

return Scheduler
\end{luacode}
\end{listing}

\chapter{Representation of the TrafficLightController state machine in Lua}
\label{app:traffic_light}

\begin{luacode}
StateMachine = require "state_machine"
Timer = require "timer"
Event = require "event"

local S0, S1, S2, S3, S4, S5 = "S0", "S1", "S2", "S3", "S4", "S5"
local YELLOW_DELAY = 3000000
local PEDESTRIAN_TIME = 10000000
local SAFE_TIME = 1000000

TrafficLightController = StateMachine:new()

TrafficLightController.events = {
	PEDESTRIAN_BUTTON_PRESSED = 1,
	YELLOW_TIMER_EXPIRED = 2,
	PEDESTRIANS_GO = 3,
	PEDESTRIAN_TIMER_EXPIRED = 4,
	CARS_GO = 5,
}

function TrafficLightController:new(id, scheduler)
	local o = {}
	setmetatable(o, { __index = self})
	o.data = {}
	o.data.id = id
	o.data.current_state = S0
	o.scheduler = scheduler
	scheduler:add_state_machine(o)
	return o
end

function TrafficLightController:fire()
	while(true) do
		local event = self.scheduler:get_active_event()
		local current_state = self:get_state()

		if event:type() == self.TERMINATE_SELF then
			break

		elseif current_state == S0 then
			if event:type() == self.events.PEDESTRIAN_BUTTON_PRESSED then
				print("Car light set to yellow.")
				local new_event = Event:new(self:id(), self.events.YELLOW_TIMER_EXPIRED)
				self.scheduler:add_timer(Timer:new(YELLOW_DELAY, self:id(), new_event))
				self:set_state(S1)
				coroutine.yield(StateMachine.EXECUTE_TRANSITION)

			else
				coroutine.yield(StateMachine.DISCARD_EVENT)
			end

-- continued on next page
\end{luacode}

\begin{listing}[htp]
\begin{luacode}
		elseif current_state == S1 then
			if event:type() == self.events.YELLOW_TIMER_EXPIRED then
				print("Car light set to red.")
				local new_event = Event:new(self:id(), self.events.PEDESTRIANS_GO)
				self.scheduler:add_timer(Timer:new(SAFE_TIME, self:id(), new_event))
				self.set_state(S2)
				coroutine.yield(StateMachine.EXECUTE_TRANSITION)

			else
				coroutine.yield(StateMachine.DISCARD_EVENT)
			end
			
		elseif current_state == S2 then
			if event:type() == self.events.PEDESTRIANS_GO then
				print("Pedestrian light set to green.")
				local new_event = Event:new(self:id(), self.events.PEDESTRIAN_TIMER_EXPIRED)
				self.scheduler:add_timer(Timer:new(PEDESTRIAN_TIME, self:id(), new_event))
				self:set_state(S3)
				coroutine.yield(StateMachine.EXECUTE_TRANSITION)

			else
				coroutine.yield(StateMachine.DISCARD_EVENT)
			end

		elseif current_state == S3 then
			if event:type() == self.events.PEDESTRIAN_TIMER_EXPIRED then
				print("Pedestrian light set to red.")
				local new_event = Event:new(self:id(), self.events.CARS_GO)
				self.scheduler:add_timer(Timer:new(SAFE_TIME, self:id(), new_event))
				self:set_state(S4)
				coroutine.yield(StateMachine.EXECUTE_TRANSITION)

			else
				coroutine.yield(StateMachine.DISCARD_EVENT)
			end

		elseif current_state == S4 then
			if event:type() == self.events.CARS_GO then
				print("Car light set to yellow.")
				local new_event = Event:new(self:id(), self.events.YELLOW_TIMER_EXPIRED)
				self.scheduler:add_timer(Timer:new(YELLOW_DELAY, self:id(), new_event))
				self:set_state(S5)
				coroutine.yield(StateMachine.EXECUTE_TRANSITION)

			else
				coroutine.yield(StateMachine.DISCARD_EVENT)
			end

		elseif current_state == S5 then
			if event:type() == self.events.YELLOW_TIMER_EXPIRED then
				print("Car light set to green.")
				self:set_state(S0)
				coroutine.yield(StateMachine.EXECUTE_TRANSITION)
			
			else
				coroutine.yield(StateMachine.DISCARD_EVENT)
			end

		else
			coroutine.yield(StateMachine.DISCARD_EVENT)
		end
	end
end

return TrafficLightController
\end{luacode}
\end{listing}

\begin{listing}
\begin{luacode}


\end{luacode}
\end{listing}

\begin{listing}
\begin{luacode}


\end{luacode}
\end{listing}

\end{appendices}