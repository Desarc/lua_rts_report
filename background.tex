\chapter{Background}
\label{chp:background}

\section{The Lua programming language}
\label{sec:lua_language}

The Lua programming language is a dynamic, multi-paradigm scripting language developed at the Pontifical Catholic University of Rio de Janeiro. Some key properties of Lua:

\begin{itemize}
	\item TODO: rewrite this
	\item Dynamically typed
	\item Interpreted scripting language
	\item Simple but powerful syntax, meta-mechanisms
	\item Simple and well-documented API for integration with other languages
	\item Collaborative multithreading
	\item Lua VM is written in "clean C"
	\item Compiled to bytecode, to be run in the Lua VM
	\item Register-based VM
	\item Incremental mark-and-sweep garbage collector
	\item Distributed under the very liberal MIT license \cite{website:lua_license}
\end{itemize}

While Lua is designed as an extension scripting language mainly for C, it is still possible to execute standalone Lua programs through the Lua interpreter \cite[ch. 7]{book:lua_reference_manual}.

Lua is the most popular interpreted scripting language for games \cite{website:the_engine_survey}.

\section{State machine -based systems}


\section{Embedded systems and M2M}