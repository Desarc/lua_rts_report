\chapter{Initial Analysis}
\label{ch:initial_analysis}
This chapter describes the analysis done prior to the experimental work, comparing Lua to other programming languages and discussing the possible use of Lua in state machine applications.

\section{Lua Compared to Other Programming Languages}
\label{sec:lua_compared}
When choosing a programming language for a certain application area, it is natural to compare it with available options. Depending on the task at hand, other languages may provide particular advantages or disadvantages for the development. The programming languages discussed in this section are either similar to Lua, or relevant with respect to embedded systems or state machine applications. The languages are compared based primarily on their most relevant aspects.

\subsection{Python}
\label{sec:lua_vs_python}
Like Lua, Python is a multi-paradigm programming language frequently used for scripting and embedding in other applications. There are however some significant differences between the two, and the official Lua wiki has a list (with some discussion) of these differences. It also sums them up nicely with the following line: \emph{``Assuming you have the luxury of a fast CPU, virtual memory, and hard disk storage, the vast library resources of Python could help get your project completed sooner. If you don't have those luxuries, Python is not an option as it is quite large.''}~\cite{website:lua_wiki_python}. Additionally, Lua is generally faster than Python, one exception being numeric computing with integers~\cite{website:lua_perl_python_vs}, which is one of Lua's major performance weaknesses.

\noindent
Since we don't have the luxuries of fast \glspl{cpu} and lots of memory in embedded systems, Lua is clearly the better choice for such applications.

\subsection{Perl}
\label{sec:lua_vs_perl}
Like with Python, the official Lua wiki has a list with some of the advantages and disadvantages for using Lua over Perl~\cite{website:lua_wiki_perl}. The differences are similar to those of Python in many areas, and this also applies to performance~\cite{website:lua_perl_python_vs}. In the context of this project, the most important points are:

\begin{itemize}
	\item{Memory footprint:} While the Lua \gls{vm} is very small, Perl requires a lot more \gls{rom} (about 1.1MB) which makes it difficult to use in embedded systems.
	\item{Portability:} Lua is based on \gls{clean-c} and thus very portable, while Perl is more difficult to adapt to new platforms.
	\item{Performance:} Lua is generally faster than Perl, which is more heavyweight. However, for some types of operations like integer arithmetic and text processing, Perl performs better than Lua.
	\item{Multitasking:} In this category, Perl is actually the winner, providing both preemptive and non-preemptive multitasking, while standard Lua only provides non-preemptive.
	\item{Libraries:} Like Python, Perl has a much larger base of available libraries than Lua.
\end{itemize}

\noindent
Overall, this makes Lua a much more suitable language for embedded systems.

\subsection{Ruby}
\label{sec:lua_vs_ruby}
Comparing Lua to Ruby is slightly less straightforward than comparing to Perl or Python. A short discussion on the official Lua wiki offers some insight, mentioning that Lua is easier to embed, at least until the release of MRuby~\cite{website:lua_wiki_ruby}. There is also some indication that Lua's performance is much better, but this apparently also depends on the future MRuby release. A more concrete evaluation of Ruby versus Lua performance shows that Lua is in fact faster than Ruby on average, but not for all cases~\cite{website:computer_language_benchmarks_game}. Also, Ruby seems to do better on memory management. These properties will also depend on which implementation is considered; Ruby has many implementations for various platforms and \glspl{vm}, and for Lua one might choose LuaJIT for increased performance. Looking at the reference implementation of Ruby which is \gls{mri}, we see that the binary image of its latest version (2.0) is about 17MB, which is quite big, especially compared to Lua's 227KB.

\noindent
Overall, Ruby seems like a lesser alternative for use in embedded systems, at least until MRuby is released.

\subsection{Java}
\label{sec:lua_vs_java}
While the previous programming languages discussed have notable similarities to Lua, Java and Lua are likely more different than alike. The following list attempts to cover the most significant differences in the context of this project.

\begin{itemize}
	\item Lua and Java use quite different syntax and coding styles. However, they are both high-level languages, and it is difficult to decide which would be easier to learn and use for less experienced programmers. This likely also depends on what the programmer is trying to accomplish. It should be noted that Java has a much larger community, so it is likely easier to find help and documentation when working with Java.
	\item With the large Java community comes a lot of libraries and modules. While Lua offers quite a few third-party libraries for common functionality, the amount of existing Java libraries is vast. This means that in many cases, the application developer will be able to find tested and well-documented third-party libraries doing at least part of the job, greatly simplifying the development process.
	\item Lua and Java are both very portable, and given a working interpreter/\gls{vm}, programs should run on any platform. However, while the Lua interpreter is open source and relatively simple to port~\cite{chapter:porting_lua_microcontroller}, the Java \gls{vm} is proprietary, closed source, and owned by the Oracle Corporation, meaning you're generally stuck with the platforms they want to support.
	\item Where Lua is considered to be lightweight, Java is not. As briefly discussed in Sect.~\ref{sec:reactive_blocks}, even a specialized Java \gls{vm} will only run on systems with a lot of memory available, several times the amount required for the Lua interpreter.
	\item Java programs are generally faster than Lua programs~\cite{website:computer_language_benchmarks_game}. This is likely because of several reasons, but the fact that Java is always compiled to bytecode is likely significant. However, considering how much faster the LuaJIT-interpreter is than the standard interpreter~\cite{website:luajit_performance}, it is likely that for a lot of programs, Lua can match the performance of Java.
\end{itemize}

\noindent
Given these properties, the conclusion for the comparison between Java and Lua is similar to that of the Lua-Python comparison. If we have an abundance of memory at our disposal, and the interpreter is ported to our platform, the better choice is in most cases Java. However, in embedded environments, these requirements are not likely to be fulfilled.

\subsection{C}
\label{sec:lua_vs_c}
When working with embedded systems on smaller devices, it is very natural to consider using C. C is by far the most commonly used programming language for embedded systems~\cite{phd:dunkels}, because of the control it gives over lower level entities like \gls{io} devices and memory, combined with good performance and (if the program is well-designed) low memory use. This makes the effort required to port a program to a new architecture relatively low, because you don't have to rewrite a whole \gls{vm}. However, coding in C is significantly more tedious, as we have to handle things memory allocation. We would like to provide application development at a ``higher level'', where these things are handled automatically, making the development process a lot quicker and easier (especially for less experienced programmers).

\noindent
Since Lua is written in C, a Lua state machine application will obviously be largely C-based anyway, but this will (hopefully) be hidden by the Lua implementation. The performance loss is obviously a good argument in favor of C, but with the available (and simple) C \gls{api} in Lua, some of the worse cases could potentially be written as libraries in C code, diminishing this loss.

\section{Lua and State Machine Systems}
\label{sec:lua_and_state_machines}
This section contains a fairly superficial analysis of how suitable or not the Lua programming language may be for state machine systems, based on its properties (some of which are summarized in Sect.~\ref{sec:lua_language}).

\subsection{The Advantages of State Machine Systems in Lua}
\label{sec:lua_advantages}
As a starting point, some of the apparent advantages of developing state machine applications in Lua are mentioned.

\paragraph{Speed}
Lua is considered to be one of the fastest available scripting languages, and independent benchmarks support this claim~\cite{website:computer_language_benchmarks_game}. Speed versus memory usage is a common trade-off in computer systems, and this is highly relevant also for embedded real-time systems, where both memory and speed have restricting limits. It is unlikely that Lua performance can (on average) match that of a fully compiled language like Java or C++ on any architecture, but relatively good performance combined with a low memory cost and low complexity makes Lua a possible middle-option.

\noindent
The performance of Lua may be further increased with LuaJIT,\footnote{http://luajit.org/} however at a slightly increased memory cost. LuaJIT is an independently maintained \gls{jit} for Lua, whose purpose is to increase Lua performance by compiling (and caching) Lua code to machine code (as opposed to bytecode, which is what the standard Lua interpreter does). LuaJIT has been shown to increase Lua performance by significant factors for single threaded programs on various processors~\cite{website:luajit_performance}, but comes at the cost of decreased portability.

\paragraph{Embedding}
In addition to the Lua interpreter itself being portable, Lua is easy to embed into applications written in other languages. Similarly, Lua may be extended with libraries written in other languages. Lua is actually designed to be an extension language, and this area also accounts for most of its use~\cite{website:where_lua_is_used}. Lua is particularly refined for working with C, providing a robust and flexible C \gls{api}~\cite[Ch.24]{book:programming_in_lua_first}. This means that functionality that may be inconvenient to implement in Lua (e.g. \gls{io} handling on a microcontroller) can be added through a C library with relative ease. Additionally, in the context of a state machine system, it might be possible to implement a runtime support system in C for efficiency and control, including a Lua interpreter for state machine definitions on a higher level.

\paragraph{Sandboxing}
In Lua, functions are first-class values~\cite[Ch.2.2]{manual:lua_reference_manual}. Additionally, references to the standard library functions are stored in variables in the Lua global environment. Whenever a globally defined function is called, Lua looks it up in the global environment table. If we wish to restrict access to certain functions, the table entry may be replaced by a different function, performing some checking before access to the actual (renamed) library function is given. In a state machine system, this could be useful if we want to allow the execution of arbitrary state machines, and allows us to implement safety and security with relative ease. Additionally, memory and \gls{cpu} usage can be limited by use of the standard Lua debugging library~\cite[Ch.6.10]{manual:lua_reference_manual}.

\paragraph{Collaborative multitasking}
Lua supports multitasking in the form of coroutines, which is collaborative (non-preemptive). Coroutines do not run in parallel on a processor, but explicitly yield control of the processor to each other. The purpose of coroutines is to offer multitasking without the complexity of interrupts, race conditions and deadlocks, as well as structuring programs in clearer and more restrictive ways for better internal control. Each coroutine is an independent ``thread'' with its own stack, but memory may be shared between threads. This is ideal for a state machine based application with run-to-completion semantics: the scheduler may yield control of the processor to a state machine, who will continue execution based on the state of its own stack. When an event has been processed and a transition completed, the state machine simply resumes control of the processor to the scheduler.

\paragraph{Complexity}
One of the goals of Lua is to be simple, without being simplistic~\cite{article:the_implementation_of_lua}. It is fairly easy for less experienced programmers to read and use Lua, allowing users to add functionality without having to learn all the underlying quirks. At the same time, the Lua meta-mechanisms allow for more complex and powerful use of the language, resulting in support for additional paradigms (like object-oriented programming). This could imply that getting started with Lua is easy, while mastering its concepts might be more difficult. In any case, given a working interpreter for a specific platform, it is likely easier for the less experienced developer to get started on application development, compared to working with a language like C.

\subsection{The Disadvantages State Machine Systems in Lua}
\label{sec:lua_disadvantages}
While there are many arguments that speak in favor of exploring Lua for state machine based systems, there are also some apparent limitations. These may be specific to some particular cases, or more general and pervasive, and are briefly discussed below.

\paragraph{Multitasking}
Lua offers non-preemptive multitasking, which is probably sufficient in most cases for an extension language, and for many types of state machine systems. However, if we want a general pure Lua state machine runtime support system, preemptive multitasking may to some degree be necessary. This however depends on the kind of semantics we want to allow in our system (such as preempting ongoing transitions), and whether or not we want to allow full control of processing time for arbitrary state machines.

\paragraph{Timers}
A first glance at the Lua reference manual indicates that Lua offers very limited facilities for time measurement. Standard Lua allows for \emph{second}-based timestamps as the finest granularity, which may be sufficient for the intended use of Lua as an extension language, but becomes a problem if one wants to implement real-time systems fully in Lua. Ideally, at least millisecond granularity should be available. While there are third-party libraries that solve this issue, there are no options for the standard core Lua.

\paragraph{Software libraries}
One of the goals of Lua is to provide a core that is as small as possible, and thus stripped of most non-essential parts. This means that standard Lua only offers basic Lua functionality~\cite[Ch.6]{manual:lua_reference_manual}, and anything else must be added through extra libraries. While there are many different third-party libraries available for Lua, these tend to have less portability than the core itself, limiting our options if we want to run Lua on anything but the most common platforms. Additionally, Lua libraries are far less in number and diversity than those of its competitors, like for example Python.

\section{Lua and Embedded Systems}
\label{sec:lua_and_embedded}
This section contains a brief analysis of Lua with respect to how suitable the language may be for use in embedded devices. High-level languages like Lua are less commonly used in embedded systems, because they are often difficult to port and additionally require an abundance of resources (particularly memory). Lua has already gained a minor foothold in embedded systems, as described in Sect.~\ref{sec:lua_in_embedded}, and this section attempts to explain why.

\paragraph{Memory footprint}
The standard Lua core was purposefully designed to be minimalistic, offering only a base set of functions~\cite{article:the_implementation_of_lua}. As a result of this, the extra memory used, by either adding Lua to your application (extending) or running the code through the Lua interpreter, is minimal. The Lua interpreter built with the standard libraries requires only a total of 227KB \gls{rom} on Linux. Without the standard libraries, the memory footprint is actually less than 100KB~\cite{website:lua_about}. This is an important property, especially when working with embedded systems, where memory capabilities are usually severely restricted. Lua thus allows for high-level application scripting without adding much overhead.

\paragraph{Portability}
The Lua \gls{vm} is written in \gls{clean-c}, which means the interpreter can be compiled and built out-of-the-box on any platform with a standard C compiler~\cite{article:the_implementation_of_lua}, and may similarly be adapted and cross-compiled also to run on even more platforms. The Lua core has been deliberately designed to allow for portability to initially ``unsupported'' platforms with minimal changes to its code~\cite{chapter:porting_lua_microcontroller}. Lua scripts are compiled to bytecode and run completely within the \gls{vm}, so with a working Lua interpreter, any standard Lua program will work for the given platform.
